% This example is meant to be compiled with lualatex or xelatex
% The theme itself also supports pdflatex
\PassOptionsToPackage{unicode}{hyperref}
\documentclass[aspectratio=1610, 12pt]{beamer}

% Warning, if another latex run is needed
% \usepackage[aux]{rerunfilecheck}

% just list chapters and sections in the toc, not subsections or smaller
\setcounter{tocdepth}{1}

%------------------------------------------------------------------------------
%------------------------------ Fonts, Unicode, Language ----------------------
%------------------------------------------------------------------------------
\usepackage{fontspec}
\defaultfontfeatures{Ligatures=TeX}  % -- becomes en-dash etc.

% german language
\usepackage{polyglossia}
\setdefaultlanguage{german}

% for english abstract and english titles in the toc
\setotherlanguages{english}

% intelligent quotation marks, language and nesting sensitive
\usepackage[autostyle]{csquotes}

% microtypographical features, makes the text look nicer on the small scale
\usepackage{microtype}

%------------------------------------------------------------------------------
%------------------------ Math Packages and settings --------------------------
%------------------------------------------------------------------------------

\usepackage{amsmath}
\usepackage{amssymb}
\usepackage{mathtools}
\usepackage{bbold}

% Enable Unicode-Math and follow the ISO-Standards for typesetting math
\usepackage[
  math-style=ISO,
  bold-style=ISO,
  sans-style=italic,
  nabla=upright,
  partial=upright,
]{unicode-math}
\setmathfont{Latin Modern Math}

% nice, small fracs for the text with \sfrac{}{}
\usepackage{xfrac}


%------------------------------------------------------------------------------
%---------------------------- Numbers and Units -------------------------------
%------------------------------------------------------------------------------

\usepackage[
  locale=DE,
  separate-uncertainty=true,
  per-mode=symbol-or-fraction,
]{siunitx}
\sisetup{math-micro=\text{µ},text-micro=µ}
% \sisetup{tophrase={{ to }}}
%------------------------------------------------------------------------------
%-------------------------------- tables  -------------------------------------
%------------------------------------------------------------------------------

\usepackage{booktabs}       % \toprule, \midrule, \bottomrule, etc

%------------------------------------------------------------------------------
%-------------------------------- graphics -------------------------------------
%------------------------------------------------------------------------------

\usepackage{graphicx}
%\usepackage{rotating}
\usepackage{grffile}
\usepackage{tikz}
\usepackage{circuitikz}
\usepackage{tikz-feynman}
\usepackage{subcaption}

% allow figures to be placed in the running text by default:
\usepackage{scrhack}
\usepackage{float}
\floatplacement{figure}{htbp}
\floatplacement{table}{htbp}

% keep figures and tables in the section
\usepackage[section, below]{placeins}

% smileys
\usepackage{MnSymbol,wasysym}

%------------------------------------------------------------------------------
%---------------------- customize list environments ---------------------------
%------------------------------------------------------------------------------

\usepackage{enumitem}
\usepackage{listings}
\usepackage{hepunits}

\usepackage{pdfpages}
%------------------------------------------------------------------------------
%------------------------------ Bibliographie ---------------------------------
%------------------------------------------------------------------------------

\usepackage[
  backend=biber,   % use modern biber backend
  autolang=hyphen, % load hyphenation rules for if language of bibentry is not
                   % german, has to be loaded with \setotherlanguages
                   % in the references.bib use langid={en} for english sources
]{biblatex}
\addbibresource{references.bib}  % the bib file to use
\DefineBibliographyStrings{german}{andothers = {{et\,al\adddot}}}  % replace u.a. with et al.


% Load packages you need here
% \usepackage{polyglossia}
% \setmainlanguage{german}

\usepackage{csquotes}


% \usepackage{amsmath}
% \usepackage{amssymb}
% \usepackage{mathtools}

\usepackage{hyperref}
\usepackage{bookmark}

% load the theme after all packages

\usetheme[
  showtotalframes, % show total number of frames in the footline
]{tudo}

% Put settings here, like
\unimathsetup{
  math-style=ISO,
  bold-style=ISO,
  nabla=upright,
  partial=upright,
  mathrm=sym,
}

% \setbeamertemplate{itemize item}{\scriptsize$\blacktriangleright$}
% \setbeamertemplate{itemize subitem}{\scriptsize$\blacktriangleright$}

%Titel:
\title{Understanding the alignment of LHCb's SciFi Tracker}
%Autor
\author[N.Breer]{Nils Breer*, Sophie Hollitt, Johannes Albrecht}
%Lehrstuhl/Fakultät
\institute{TU Dortmund, Fakultät Physik}
%Titelgrafik muss ich einfueren!!!
%\titlegraphic{\includegraphics[width=0.3\textwidth]{content/Bilder/interferenz.jpg}}
\date{08.03.2023}

\begin{document}
\maketitle

\begin{frame}\frametitle{Overview and Motivation}
$\textbf{Motivation}$
  \begin{itemize}
    \item $\bullet$\, Studying permormance of different alignments on run 256145 data
    \item \to\, unexpected different results!
    \item \to\, analysis of individual quarters
  \end{itemize}
$\textbf{Overview}$
  \begin{itemize}
    \item $\bullet$\, The SciFi Detector Upgrade
    \item $\bullet$\, Alignment how to
    \item $\bullet$\, Analysis of SciFi quarters in different alignment versions
  \end{itemize}
\end{frame}

\begin{frame}\frametitle{The Scintillating Fibre Tracker}
  \begin{columns}
    \begin{column}[c]{0.48\textwidth}
      \begin{figure}
        \subfloat[]{%
          \includegraphics[clip,width=0.65\columnwidth]{logos/LHCb.png}%
        }

        \subfloat[]{%
          \includegraphics[clip,width=0.65\columnwidth]{logos/lumi_lhcb.png}%
        }

      \end{figure}
    \end{column}
    \begin{column}[c]{0.48\textwidth}
      \begin{itemize}
        \item $\bullet$\, Higher luminosity
        \begin{itemize}
          \item $\bullet$\, detector must operate well with expected radiation damage
        \end{itemize}
        \item $\bullet$\, detector readout electronics need to operate at 40 MHz, 25ns usable time per collision
        \item $\bullet$\, tracking efficiency and hit detection improvements aim for about 98\% hit detection rate
      \end{itemize}
    \end{column}
  \end{columns}
\end{frame}

\begin{frame}\frametitle{The Scintillating Fibre Tracker}
  \begin{columns}
    \begin{column}[c]{0.48\textwidth}
      \begin{figure}
        \includegraphics[width=0.9\textwidth]{logos/scifi.png}
        \caption{Visualization of the SciFi tracking stations.}
      \end{figure}
    \end{column}
    \begin{column}{0.48\textwidth}
      \begin{itemize}
        \item $\bullet$\, single detector type vs. IT + OT
        \item $\bullet$\, less timing information needed for readout
        \item $\bullet$\, less detector material
        \begin{itemize}
          \item $\bullet$\, less multiple scattering and material interactions
        \end{itemize}
        \item $\bullet$\, SiPM technology improvements yield better resolution and speed
      \end{itemize}
    \end{column}
  \end{columns}
\end{frame}

\begin{frame}\frametitle{What is Alignment?}
  \begin{columns}
    \begin{column}[c]{0.48\textwidth}
      \input{detector.tex}
    \end{column}
    \begin{column}[c]{0.48\textwidth}
      \begin{itemize}
        \item $\bullet$\, top: ideal detector, bottom: physical detector
        \item $\bullet$\, Surveys are used to find the rotation and position of each detector component
        \item $\bullet$\, Are used as starting positions for software alignment (this talk!)
      \end{itemize}
    \end{column}
  \end{columns}
\end{frame}

\begin{frame}\frametitle{Alignment: track fits with the Kalman Filter}
  \begin{columns}
    \begin{column}[c]{0.48\textwidth}
      \begin{figure}
        \centering
        \includegraphics[width=0.9\textwidth]{logos/kalman.png}
        \caption{Alignment with Kalman Filter.}
      \end{figure}
    \end{column}
    \begin{column}[c]{0.48\textwidth}
      \begin{itemize}
        \item $\bullet$\, Minimise $\chi^2$ with respect to the track parameters for the track fit
        \item $\bullet$\, Minimise $\chi^2$ with respect to the alignment parameters $\alpha$ during the alignment
        \item $\bullet$\, Update the alignment constants $\alpha$ and repeat until convergence criterium for $\chi^2$ is reached
      \end{itemize}
    \end{column}
  \end{columns}
\end{frame}

\begin{frame}\frametitle{Alignment versions in use}
  \begin{columns}
    \begin{column}[c]{0.33\textwidth}
      \begin{itemize}
        \item V1:
        \item use full length Modules
        \item alignable degrees of freedom: Tx Rz (x translation, rotation around z \to beam pipe axis)
      \end{itemize}
    \end{column}
    \begin{column}[c]{0.33\textwidth}
      \begin{itemize}
        \item $\text{low} \mu$:
        \item use half modules
        \item uses VELO alignment on run 256145 data
        \item $\mu =$ look it up
      \end{itemize}
    \end{column}
    \begin{column}[c]{0.33\textwidth}
      \begin{itemize}
        \item V2:
        \item newest alignment version
        \item half modules (top half and bottom half)
        \item uses newest time alignment
        \item utilizes VELO alignment from run 256145
        \item $\mu \approx 2.26$ (value taken from run database)
      \end{itemize}
    \end{column}
  \end{columns}
\end{frame}

\begin{frame}\frametitle{Why analyse the quarters separately?}
  \begin{itemize}
    \item $\bullet$\, perfomance in each quarter might be very different from one another
    \item $\bullet$\, \to $\chi^2$ per layer might be different from $\chi^2$ per quarter
    \item $\bullet$\, v2 alignment shows improvements from v1 alignment but not across the whole SciFi
    \item $\bullet$\, find and resolve possible issues is easier
  \end{itemize}
\end{frame}

\begin{frame}\frametitle{Hit distribution per quarter in V1 and V2 alignment}
  \begin{itemize}
    \item $\bullet$\, V1(left)- and V2(right) alignment on 20000 events with run 256145 data
    \item $\bullet$\, C-side: negative x direction, A-side: positive x
    \item $\bullet$\, plotted is x-coordinate against number of hits in each quarter coded by colour.
    \item $\bullet$\, 9 minimum hits per quarter (solid lines), minimum hits (dashed lines)
  \end{itemize}
  \begin{figure}
    \subfloat[V1 alignment]{%
      \includegraphics[width=0.45\textwidth]{compareHitNums/DataSeedsTupled_node_X_All_run256145_V1.pdf}%
    }
    \subfloat[V2 alignment]{%
      \includegraphics[width=0.45\textwidth]{compareHitNums/DataSeedsTupled_node_X_All_run256145_V2.pdf}%
    }
    % \caption{Hits on tracks in x-direction with run 256145 data on 20000 events using 9 minimum hits against 11 minimum hits.}
  \end{figure}
\end{frame}

\begin{frame}\frametitle{Summary of Metrics from alignments in Quarter 0}
  This hints that something is not right in Q0
  \begin{columns}
    \begin{column}[c]{0.48\textwidth}
      \begin{figure}
        \centering
        \includegraphics[width=0.9\textwidth]{2023-mar-9-DPG/chi2_per_ndof_Q0.pdf}
        \caption{track $\chi^2$ per dof comparing each alignment for Quarter 0.}
      \end{figure}
    \end{column}
    \begin{column}{0.48\textwidth}
      \begin{figure}
        \includegraphics[width=0.9\textwidth]{2023-mar-9-DPG/RMSResidualQuarters_Q0.pdf}
        \caption{Residual in each module for each alignment in Quarter 0.}
      \end{figure}
    \end{column}
  \end{columns}
\end{frame}

\begin{frame}\frametitle{Summary of Metrics from alignments in Quarter 1}
  \begin{columns}
    \begin{column}[c]{0.48\textwidth}
      \begin{figure}
        \centering
        \includegraphics[width=0.9\textwidth]{2023-mar-9-DPG/chi2_per_ndof_Q1.pdf}
        \caption{track $\chi^2$ per dof comparing each alignment for Quarter 1.}
      \end{figure}
    \end{column}
    \begin{column}{0.48\textwidth}
      \begin{figure}
        \includegraphics[width=0.9\textwidth]{2023-mar-9-DPG/RMSResidualQuarters_Q1.pdf}
        \caption{Residual in each module for each alignment in Quarter 1.}
      \end{figure}
    \end{column}
  \end{columns}
\end{frame}

\begin{frame}\frametitle{Summary of Metrics from alignments in Quarter 2}
  \begin{columns}
    \begin{column}[c]{0.48\textwidth}
      \begin{figure}
        \centering
        \includegraphics[width=0.9\textwidth]{2023-mar-9-DPG/chi2_per_ndof_Q2.pdf}
        \caption{track $\chi^2$ per dof comparing each alignment for Quarter 2.}
      \end{figure}
    \end{column}
    \begin{column}{0.48\textwidth}
      \begin{figure}
        \includegraphics[width=0.9\textwidth]{2023-mar-9-DPG/RMSResidualQuarters_Q2.pdf}
        \caption{Residual in each module for each alignment in Quarter 2.}
      \end{figure}
    \end{column}
  \end{columns}
\end{frame}

\begin{frame}\frametitle{Summary of Metrics from alignments in Quarter 3}
  \begin{columns}
    \begin{column}[c]{0.48\textwidth}
      \begin{figure}
        \centering
        \includegraphics[width=0.9\textwidth]{2023-mar-9-DPG/chi2_per_ndof_Q3.pdf}
        \caption{track $\chi^2$ per dof comparing each alignment for Quarter 3.}
      \end{figure}
    \end{column}
    \begin{column}{0.48\textwidth}
      \begin{figure}
        \includegraphics[width=0.9\textwidth]{2023-mar-9-DPG/RMSResidualQuarters_Q3.pdf}
        \caption{Residual in each module for each alignment in Quarter 3.}
      \end{figure}
    \end{column}
  \end{columns}
\end{frame}

\begin{frame}\frametitle{$\Chi^2$ against $\phi$ angle distribution in V2 alignment}
  \begin{columns}
    \begin{column}[c]{0.48\textwidth}
      \begin{figure}
        \centering
        \includegraphics[width=0.9\textwidth]{compareHitNums/chi2ProbVsPhi_RMSResidualQuarters_v2.pdf}
        \caption{$\Chi^2$ against $\phi$ distribution for each quarter for V2 alignment.}
      \end{figure}
    \end{column}
    \begin{column}[c]{0.48\textwidth}
      \begin{itemize}
        \item $\bullet$\, $\Chi^2$ against $\phi$ distribution for each quarter in V2
        \item $\bullet$\, information of layers are combined for each quarter
        \begin{itemize}
          \item \to\, information of problematic layer in given quarter hidden
        \end{itemize}
        \item $\bullet$\, aim: flat distribution across all angles
        \item $\bullet$\, A-side quarters (Q1: blue, Q3: black) quite flat
        \item $\bullet$\, C-side quarters (Q0: green, Q2: red) have small $\Chi^2$ around 0
      \end{itemize}
    \end{column}
  \end{columns}
\end{frame}

\begin{frame}\frametitle{Track residuals in bottom half SciFi quarters}
  \begin{itemize}
    \item $\bullet$\, left: Quarter 0, right: Quarter 1
    \item $\bullet$\, use cuts on LayerID to extract information about each layer
    % \item $\bullet$\,
  \end{itemize}
  \begin{figure}
    \subfloat[Quarter 0]{%
      \includegraphics[width=0.45\textwidth]{2023-feb-07/mean_residuals_Q0_allLayers.pdf}%
    }
    \subfloat[Quarter 1]{%
      \includegraphics[width=0.45\textwidth]{2023-feb-07/mean_residuals_Q1_allLayers.pdf}%
    }
  \end{figure}
\end{frame}

\begin{frame}\frametitle{Track residuals in top half SciFi quarters}
  \begin{itemize}
    \item $\bullet$\, left: Quarter 2, right: Quarter 3
    \item $\bullet$\, use cuts on LayerID to extract information about each layer
    % \item $\bullet$\,
  \end{itemize}
  \begin{figure}
    \subfloat[Quarter 2]{%
      \includegraphics[width=0.45\textwidth]{2023-feb-07/mean_residuals_Q0_allLayers.pdf}%
    }
    \subfloat[Quarter 3]{%
      \includegraphics[width=0.45\textwidth]{2023-feb-07/mean_residuals_Q1_allLayers.pdf}%
    }
  \end{figure}
\end{frame}

\begin{frame}\frametitle{Conclusion}
    \begin{columns}
      \begin{column}[c]{0.48\textwidth}
        \begin{figure}
          \centering
          \includegraphics[width=0.9\textwidth]{2023-mar-9-DPG/meanResidual_AlignTracks_weighted.pdf}
          \caption{mean Residual per layer weighted with quarter hits.}
        \end{figure}
      \end{column}
      \begin{column}{0.48\textwidth}
        \begin{itemize}
          \item mean residual per quarter weighted: $
              \bar{Res_{Q}} = \sum_{\text{layer}, \text{quarter}} \frac{\text{hits quarter of layer}}{\text{hits layer}}$
          \item goal: residual around 0 per layer
          \item V2: quite good except second C-frame in T2
          \item V1: everywhere worse than V2
          \item low $\mu$: quite ok except for back T2
        \end{itemize}
      \end{column}
    \end{columns}
  \to\, V2 best performing alignment version for now, but still uses half modules
  \to\, long modules as in the physical SciFi preferred in the long run
\end{frame}

\begin{frame}\frametitle{Track residuals in top half SciFi quarters}
  \begin{itemize}
    \item $\bullet$\, V1: left, V2, right
    \item $\bullet$\, Hits on tracks as XY distribution resembling SciFi Layers
    \item $\bullet$\, C-side: negative x, A-side: positive x
    \item $\bullet$\, information of all layers are combined for each quarter \to\, hard to see whats going on
  \end{itemize}
  \begin{figure}
    \subfloat[Quarter 2]{%
      \includegraphics[width=0.45\textwidth]{tuples_out/combining_2D_nodeXY_v1.pdf}%
    }
    \subfloat[Quarter 3]{%
      \includegraphics[width=0.45\textwidth]{tuples_out/combining_2D_nodeXY_v2.pdf}%
    }
  \end{figure}
\end{frame}

\begin{frame}\frametitle{Track residuals in top half SciFi quarters}
  \begin{itemize}
    \item $\bullet$\, V1: left, V2, right
  \end{itemize}
  \begin{figure}
    \subfloat[Quarter 2]{%
      \includegraphics[width=0.45\textwidth]{2023-feb-07/v1_layers/2D_nodeXY_v1_7.pdf}%
    }
    \subfloat[Quarter 3]{%
      \includegraphics[width=0.45\textwidth]{2023-feb-07/v2_layers/2D_nodeXY_v2_7.pdf}%
    }
  \end{figure}
\end{frame}

\begin{frame}\frametitle{Conclusion}
  \begin{itemize}
    \item $\bullet$\, text
  \end{itemize}
\end{frame}

\begin{frame}\frametitle{Sources}
  \begin{itemize}
    \item $\bullet$\,SciFi Conference Talk: \url{https://twiki.cern.ch/twiki/pub/LHCb/SciFiConference/fee_2018.pdf}
    \item $\bullet$\,LHCb SciFi: From performance requirements to an operational detector: \url{https://indico.cern.ch/event/1163878/}
  \end{itemize}
\end{frame}

\end{document}
